\Introduction

Сервис для агрегирования инфомарции по call-центру предназначен для наблюдения за различными актуальными
показателями работы call-центра.

Многим супервизорам или руководителям отделов требуется, время от времени, знать,
что происходит в call-центре, в каком состоянии находятся операторы, проекты или звонки.
Это знание поможет им принять правильное решение, учитывающее текущую обстановку, например,
так они смогут в случае, если нагрузка на какой-то конкретный проект сильно возросла
и операторы не справляются с поступающими вызовами,
перевести больше операторов на проект или перераспределить вызовы на другие проекты.
Разрабатываемый сервис предназначен решить проблему наблюдения в удобной форме,
предоставляя различные сводки и временные графики.

\Define{Naumen Contact Center}{полнофункциональное программное решение для построения крупных и средних контактных центров~\cite{doc:intro}}
\Define{Naumen}{российская компания, разработчик программного обеспечения, основана в 2001 году в Екатеринбурге}
Сервис будет использоваться и распространяться в составе программного продукта Naumen Contact Center.
Naumen Contact Center (ранее Naumen Phone) разрабатывается российской компанией Naumen.
Проектный офис и внедренческий центр компании находится в Москве, разработка ведётся в Екатеринбурге, Твери, Челябинске,
Санкт-Петербурге и Севастополе.
В компании действует собственный учебный и сервисный центр.
Существует дочерняя компания в Казахстане, занимающаяся работой с клиентами из Средней Азии.
В состав NCC входит коммуникационная платформа с компонентом Omni-Channel, WFM
\Abbrev{WFM}{информационная система управления проектами и прогнозирования рабочей нагрузки},
а также автоматизированные рабочие места оператора и супервизора.
Клиентами Naumen являются:
ИнтерРАО
ЕС,
Спортмастер,
Банк
Россия,
Мосэнергосбыт,
Moldtelecom.
Платформа Naumen Contact Center включена в глобальный отчет Gartner~\cite{doc:intro}.

На текущий момент на рынке представлено достаточное количество решений для реализации наблюдения за работой call-центра~\cite{other:rival}.
Но все они, либо являются частью конкурирующих программный продуктов для автоматизации call-центров,
либо интеграция с ними является не тривиальной задачей,
т.~к. для реализации наблюдения за состоянием call-центра нужно учитывать многие внутренние механизмы и протоколы.
\Abbrev{NCC}{Naumen Contact Center}
Поэтому было решено разработать собственное решение, которое учитывало бы все ньюансы работы внутри NCC\@.

Это уже не первая попытка реализации сервиса такого рода.
Предыдущая попытка закончилась с переменным успехом: сервис не обеспечивал достаточный уровень актуальности данных
на call-центрах с больших количеством операторов.
Поэтому, в случае успеха, от него планируется отказаться.

Главным аргументом успешности текущей реализации является изменение технологического стека,
если предыдущие попытки были реализованы с помощью языка Python 2.7 и сетевого фрэймворка Twisted~\cite{info:twisted}, %todo [https://twistedmatrix.com/trac/]
то текущая попытка реализации будет выполнена на языке Go.
Go предоставляет средства и возможности для разработки производительного сетевого микросервиса~\cite{youtube:go}.

\Abbrev{ПО}{программное обеспечение}
Пояснительная записка состоит из введения, постановки задачи, анализа, описания разработаного ПО, руководства пользователя и заключения.

В разделе постановки задачи описывается объект автоматизации, проблемы старой технологии и аналогов на рынке
и предъявляются требования к новой реализации, в том числе к интерфейсу и документированию.

В разделе анализа поставленной задачи приводится анализ выдвинутых требований,
описание и обоснование разработанной архитектуры,
а так же обоснование выбора инструментальных средств и интерфейса пользователя.
В этом же разделе приводится функциональная модель системы.

В разделе описания ПО, описывается внутренняя структура сервиса и механизм его работы.
Так же здесь описана схема базы данных.

В разделе руководства пользователя описан процесс установки, требуемые аппаратные ресурсы, возможные регламентные работы,
а так же порядок работы с системой.

В приложениях приведены исходные коды системы и пример конфигурационного файла.

Пояснительная записка оформлена с учетом ГОСТ 7.32--2017~\cite{gost732} и учебно-методического пособия~\cite{smp}.
