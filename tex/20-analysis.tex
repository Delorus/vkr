\chapter{Постановка задачи}
\label{cha:analysis}
%
% % В начале раздела  можно напомнить его цель
%

\section{Описание объекта автоматизации}

\subsection{Краткие сведения об объекте автоматизации}

\Abbrev{RRS}{Real-time report subsystem --- подсистема отчетов реального времени}
Объектом автоматизации сервиса для агрегирования информации по call-центру (далее RRS) является часть обязанностей супервизора,
связанных с наблюдением за текущим состоянием доверенных ему проектов и операторов.

\subsection{Сведения об условиях эксплуатации объекта автоматизации}

RRS является частью Naumen Contact Center.
NCC состоит из модулей и сервисов, каждый из которых выполняет свою функцию.
Обмен данными между компонентами NCC осуществляется по стандартным сетевым протоколам,
что позволяет физически размещать компоненты продукта на разных серверах.
Такая особенность архитектуры позволяет строить масштабируемые call-центры
для одновременного обслуживания массового количества вызовов.

NCC позволяет решать следующие основные задачи:
\begin{enumerate}
    \item организация проектов по обработке входящих обращений, в том числе:
        \begin{enumerate}
            \item обработка потока входящих телефонных вызовов;
            \item обработка потока входящих E-mail-сообщений;
            \item обработка потока входящих SMS-сообщений;
            \item обработка входящих мгновенных сообщений;
        \end{enumerate}
    \item организация исходящих проектов, включая:
        \begin{enumerate}
            \item автоматические исходящие обзвоны;
            \item E-mail рассылки;
        \end{enumerate}
    \item организация смешанных проектов, включая:
        \begin{enumerate}
            \item обработка обращений, поступивших по различным каналам связи одним и тем же оператором;
            \item обработка как входящих обращений, так и исходящих вызовов одним и тем же оператором;
        \end{enumerate}
    \item контроль работы операторов, в том числе:
        \begin{enumerate}
            \item осуществление записи разговоров;
            \item получение отчетности;
            \item управление качеством обработки вызовов.
        \end{enumerate}
\end{enumerate}

\Abbrev{PMS}{Project Management System --- система управления проектами}
Система управления проектами (PMS) представляет собой Web-ориентированную систему
управления проектами по обслуживанию обращений клиентов.
PMS включает в себя следующие основные функции:
\begin{enumerate}
    \item управление партнерами, которые выступают в качестве заказчиков на проекты по обслуживанию вызовов;
    \item формирование и ведение проектов, в том числе:
        \begin{enumerate}
            \item формирование состава участников;
            \item разработка сценариев обслуживания обращений;
            \item формирование заданий на обслуживание вызовов;
            \item контроль хода выполнения работ по проекту;
        \end{enumerate}
    \item использование встроенного программного телефона WebPhone;
    \item формирование отчетности и предоставление ее партнерам.
\end{enumerate}

\Define{NauCore}{Сервис шины управляющих сообщений, обеспечивает работу остальных сервисов NCC и их взаимодействие между собой}
PMS взаимодействует с другими сервисами через брокер сообщений NauCore посредством внутреннего протокола обмена сообщениям NCC.

NauCore запускается при загрузке операционной системы и выполняет следующие функции:
\begin{enumerate}
    \item осуществляет взаимодействие других компонентов NCC между собой;
    \item автоматически запускает и контролирует работу других телефонных сервисов. При аварийном завершении работы какого-либо сервиса пытается его перезапустить;
    \item предоставляет интерфейс управления сервисами, которые он контролирует:
        \begin{enumerate}
            \item web-интерфейс;
            \item командная строка;
        \end{enumerate}
    \item осуществляет ротацию журналов работы сервисов.
\end{enumerate}

Сервисы NauCore устанавливаются на каждый сервер NCC,
соединяются между собой и образуют общую шину обмена сообщениями.
При запуске каждый сервис NauCore автоматически запускает другие сервисы NCC
и обеспечивает их взаимодействие через общую шину.

NCC использует для связи компонент системы XML-over-TCP протокол.

\Abbrev{СУБД}{система управления базой данных}
\Define{PostgreSQL}{свободная объектно-реляционная система управления базами данных}
\Define{Oracle DB или Oracle RDBMS}{объектно-реляционная~система управления базами данных компании Oracle}
NCC в качестве базы данных использует общую СУБД, которой может быть либо PostgreSQL, либо Oracle DB.

RRS проектируется как отдельный сервис в NCC,
при этом задачу отображения данных берет на себя NCC PMS,
а сами данные нужно будет получать из NauCore\cite{Pup09}. %todo убрать
