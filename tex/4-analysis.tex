\chapter{Анализ поставленной задачи}
\label{ch:analysis}

\section{Анализ требований к ПО}

\subsection{Определение заинтересованных сторон, сбор требований} %todo какая-то дичь, надо бы переделать

Перечень заинтересованных сторон и их проблемы приведены в таблице~\ref{tab:itrppl}.

\begin{table}%[ht]
    \caption{Заинтересованные стороны и их цели}
    \begin{small}
    \begin{tabular}{|p{0.15\textwidth}|p{0.1\textwidth}|p{0.1\textwidth}|p{0.15\textwidth}   |p{0.4\textwidth}|}
        \hline
        Роль                          & ФИО            & Отдел          & Должность          & Цель (влияние) \\
        \hline
        Супервизор                    & И.И. Иванов    & --             & Супервизор         & Необходим простой и удобный инструмент для мониторинга за текущим состоянием операторов и проектов       \\
        \hline
        Руководитель отдела           & П.П. Петров    & --             & Руководитель отдела& Удобный инструмент для выявление основных метрик по колл-центру для возможности премирования         \\
        \hline
        Интегратор CRM                & А.Н. Коваленко & --             & Разработчик        & Необходима возможность интеграции с колл-центром в плане получения различных метрик         \\
        \hline
    \end{tabular}
    \end{small}
    \label{tab:itrppl}
\end{table}

Сформулируем проблемы в таблице~\ref{tab:problem}.

\begin{small}
\begin{longtable}{|p{0.3\textwidth}|p{0.5\textwidth}|}
    \caption{Проблемы заинтересованных сторон}
    \label{tab:problem}
    \\ \hline
    Наименование элемента & Содержание \\
    \hline \endfirsthead
    \subcaption{\normalsize{Продолжение таблицы~\ref{tab:problem}}}
    \\ \hline \endhead
    \hline
    \endfoot
    \hline \endlastfoot
    Проблема 1 & Невозможно узнать, чем в данный момент занимается конкретный оператор. \\
    Влияет на & Супервизор, руководитель отдела. \\
    Приводит к & Неспособность вовремя реагировать на различные ситуации, такие как отсутствие оператора на рабочем месте. \\
    Ее решение привело бы к & Лучшей эффективности call-центра. \\
    Важность и приоритет & Высокая важность, высокий приоритет. \\
    \hline
    Проблема 2 & Невозможно рассчитать различные показали эффективности call-центра. \\
    Влияет на & Супервизор, руководитель отдела. \\
    Приводит к & Сложности в расчете заработной премии. \\
    Ее решение привело бы к & Увеличении мотивации операторов и улучшении контроля за call-центром. \\
    Важность и приоритет & Высокая важность, высокий приоритет. \\
    \hline
    Проблема 3 & Отсутствие интеграции с используемой CRM\@. \\
    Влияет на & Интегратор CRM, супервизор. \\
    Приводит к & Расчету части метрик в ручном режиме и внос их в CRM, на что тратится огромное количество времени супервизоров. \\
    Ее решение привело бы к & Экономия рабочего времени супервизоров, которое они могут потратить на более важные дела. \\
    Важность и приоритет & Высока важность, высокий приоритет. \\
\end{longtable}
\end{small}

\subsection{Формулировка требований в области проблем и в области решений} %todo точно нужен этот раздел?

\subsubsection{Требования в области проблем}

\begin{itemize}
    \item Супервизоры и руководители отдела должны иметь доступ к отчетам реального времени в PMS\@.
    \item Супервизоры и руководители отдела должны иметь возможность настраивать представление отчетов по своему усмотрению, в том числе изменять позицию и какие показатели нужно выводить.
    \item Пользователь с ролью «Аналитик» должен иметь возможность просматривать статистику пройденных испытаний в экранной форме или в печатной форме.
    \item RRS должен уметь обрабатывать большое количество данных в реальном времени, с опозданием на не более 1 секунду.
    \item Супервизоры и руководители отдела должны иметь возможность настраивать периодичность обновления отчетов, но не менее чем раз в 1 секунду.
    \item Администраторы PMS должны иметь возможность изменять период очистки данных из БД.
\end{itemize}

\subsubsection{Требования в области решений}

\Define{Redis}{резидентная система управления базами данных класса NoSQL с открытым исходным кодом, работающая со структурами данных типа <<ключ -- значение>>}
\Define{NauSnitch}{подсистема сбора данных по call-центру}
\begin{itemize}
    \item В качестве БД необходимо использовать СУБД PostgreSQL или Oracle DB для данных не чувствительных ко времени и СУБД Redis для чувствительных данных.
    \item Клиентской средой является PMS\@.
    \item Для разработки NauSnitch должны применяться технологии с высокой степенью параллелизма, масштабируемости и эффективной утилизацией ресурсов.
\end{itemize}

\subsection{Анализ требований}

\begin{itemize}
    \item Система должна давать возможность настройки отображения отчетов.
    \item Система должна очищать данные в БД через настраиваемый промежуток времени.
    \item Система должна успешно функционировать при одновременной работе 900 операторов.
    \item Максимальное время опоздания обновления состояния call-центра не должно превышать 1 секунду.
    \item Система должна справляться с нагрузкой в 2--3 события за 1 мс.
    \item Система должна потреблять не более 2Гб ОЗУ.
    \item Система должна эффективно работать на ЦП с более чем 24 ядрами.
\end{itemize}

\section{Выбор инструментальных средств}

\section{Функциональная модель}

\section{Обоснование интерфейса пользователя}